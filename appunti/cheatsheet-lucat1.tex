\documentclass{article}
\usepackage{amsmath,amsfonts}

\title{\textsc{Cheatsheet di Statistica}}
\author{Luca Tagliavini}
\date{June 6, 2022}

\begin{document}

\maketitle

\begin{quote}
NOTA: i seguenti appunti sono \textbf{altamente} informali e sono pensati per essere
un supporto allo studente come ultimo ripasso prima di una prova. Contengono solo
uno scinto molto riassuntivo dei temi centrali al corso.
\end{quote}

\section{Probabilit\`a condizionata}

La probabilit\`a del verificarsi di un evento $A$ sapendo che si \`e verificando un evento $B$ \`e data da:

\begin{align*}
\mathbb{P}(A \mid B) = \frac{\mathbb{P}(A \cap B)}{\mathbb{P}(B)}
\end{align*}

\section{Regola della catena}

\begin{align*}
\mathbb{P}(A \cap B) = \mathbb{P}(A \mid B) \mathbb{P}(B)
\end{align*}

\section{Eventi indipendenti}

Due eventi $A$ e $B$ si dicono indipendenti se:

\begin{align*}
\mathbb{P}(A \cap B) = \mathbb{P}(A)\mathbb{P}(B)
\end{align*}

\section{Probabilit\`a totali}

Sia $A$ un generico evento e $B_1,\ldots,B_n$ una partizione di $\Omega$. Allora vale che:

\begin{align*}
\mathbb{P}(A) = \sum_{i=1}^n \mathbb{P}(A \cap B_i)
\end{align*}

\section{Formula di Bayes}

Siano $A,B$ due eventi tali che $\mathbb{P}(A), \mathbb{P}(B) > 0$. Allora vale:

\begin{align*}
\mathbb{P}(A \mid B) = \frac{\mathbb{P}(B \mid A)\mathbb{P}(A)}{\mathbb{P}(B)}
\end{align*}

\section{Distribuzioni semplici}
L'insieme delle distribuzioni semplici
\begin{align*}
    D_{n,k} = \{(x_1, \ldots, x_k) \mid x_i \in E, x_i \neq x_j \forall i,j \text{ t.c. } i \neq j\text{, con } |E|=n \}
\end{align*}
ha cardinalit\`a pari a
\begin{align*}
|D_{n,k}| = \frac{n!}{(n-k)!}
\end{align*}

\section{Distribuzioni con ripetizione}
L'insieme delle distribuzioni con ripetizione
\begin{align*}
    DR_{n,k} = \{(x_1, \ldots, x_k) \mid x_i \in E\text{, con } |E|=n \}
\end{align*}
ha cardinalit\`a pari a
\begin{align*}
|DR_{n,k}| = n^k
\end{align*}

\section{Combinazioni}
L'insieme delle combinazioni semplici
\begin{align*}
    C_{n,k} = \{A \subseteq E \mid |A| = k\text{, con } |E|=n \}
\end{align*}
ha cardinalit\`a pari a
\begin{align*}
|C_{n,k}| = {n \choose k}
\end{align*}

Le combinazioni sono usate per modellizzare il numero di occorrenze nel caso di estrazioni simultanee.

\section{Permutazioni}
Il numero di permutazioni di una coppia ordinata $E$ con $|E| = n$ \`e pari a 
\begin{align*}
    |P_n| = |D_{n,n}| = n!
\end{align*}

\section{Varianza e Covarianza}

La varianza \`e data da
\begin{align*}
Var(X) = \mathbb{E}[X^2] - \mathbb{E}[X]^2
\end{align*}

La covarianza \`e data da
\begin{align*}
Cov(X, Y) = \mathbb{E}[XY] - \mathbb{E}[X]\mathbb{E}[Y]
\end{align*}

La varianza gode della seguente propriet\`a:
\begin{align*}
Var(aX+b) = a^2 Var(x)
\end{align*}

\section{Probabilit\`a degli intervalli in $F_x$}

\begin{alignat*}{2}
\mathbb{P}(X = x) &= \mathbb{P}(X \leq x) - \mathbb{P}(X < x) &&= F_x(x) - F_x(x−) \\
\mathbb{P}(x < X \leq y) &= \mathbb{P}(X \leq y) - \mathbb{P}(X \geq x) &&= F_x(y) - F_x(x) \\
\mathbb{P}(x \leq X \leq y) &= \mathbb{P}(X \leq y) - \mathbb{P}(X < x) &&= F_x(y) - F_x(x-) \\
\mathbb{P}(x \leq X < y) &= \mathbb{P}(X < y) - \mathbb{P}(X < x) &&= F_x(y-) - F_x(x-) \\
\mathbb{P}(x < X < y) &= \mathbb{P}(X < y) - \mathbb{P}(X \leq x) &&= F_x(y-) - F_x(x)
\end{alignat*}

\section{Variabili aleatorie discrete}

Una variabile aleatoria discreta $X$ assume valori appartenenti al suo supporto $S_X = \{x_1, \ldots, x_n\}$.
La sua distribuzione prende la forma di
\begin{center}
\begin{tabular}{ c|c|c|c } 
 X & x_1 & \ldots & x_n \\ 
 \hline
 p_X & p_X(x_1) & \ldots & p_X(x_n)
\end{tabular}
\end{center}

La funzione di ripartizione $F_x$ \`e a tratti e della forma
\begin{align*}
F_x(x) = \begin{cases}
0 &\text{se } x < x_1 \\
p_X(x_1) &\text{se } x_1 \leq x < x_2 \\
\vdots \\
\sum_{i=1}^{n-1} p_X(x_i) &\text{se } x_{n-1} \leq x < x_{n} \\
\sum_{i=1}^n p_X(x_i) = 1 &\text{se } x \geq x_n \\
\end{cases}
\end{align*}

Il valore atteso \`e dato da:
\begin{align*}
\mathbb{E}[X] = \sum_{i=1}^n x_i p_X(x_i)
\end{align*}

\section{Variabili aleatorie continue}

Una variabile aleatoria continua $X$ assume valori appartenenti al suo supporto $S_X = [a, b]$.
La sua distribuzione prende la forma di una funzione $f_X$ con le seguenti propriet\`a:
\begin{itemize}
\item $\int_{-\infty}^{+\infty}f_X(x) dx = 1$
\item $f_X(x) > 0 \, \forall x$
\end{itemize}

La funzione di ripartizione $F_x$ \`e \textbf{continua} e della forma:
\begin{align*}
F_x(x) = \begin{cases}
0 &\text{se } x < x_1 \\
\inf_{-infty}^{x_1}f_X(x) dx &\text{se } x_1 \leq x < x_2 \\
\vdots \\
\inf_{-infty}^{x_{n-1}}f_X(x) dx &\text{se } x_{n-1} \leq x < x_n \\
\inf_{-infty}^{+infty}f_X(x) dx &\text{se } x \geq x_n \\
\end{cases}
\end{align*}

Il valore atteso \`e dato da:
\begin{align*}
\mathbb{E}[X] = \int_{-\infty}^{+\infty} f_X(x) dx
\end{align*}

\subsection{Standardizzazione}

Presa una variabile alteatoria discreta $X$ la sua forma standardizzata $Z$ si ottiene dalla formula

\begin{align*}
Z = \frac{X-\mu}{\sigma}
\end{align*}

\subsection{Teorema centrale del limite}

Preso un grande numero di esperimenti aleatori descritte da $n$ variabili aleatorie $X_1, \ldots, X_n$ si pu\`o

\begin{align*}
\overline{X_n} = \frac{X_1 + \ldots + X_n}{n} \\
\overline{Z_n} = \frac{\overline{X_n} - \mu}{\frac{\sigma}{\sqrt{n}}}
\end{align*}

Di conseguenza quando viene richiesto di calcolare $\mathbb{P}(X_1 + \ldots + X_n \leq l)$ si pu\`o segure la seguente catena di ugualianze:
\begin{align*}
\mathbb{P}(X_1 + \ldots + X_n \leq l) &= \mathbb{P}(X_n \leq \frac{l}{n})\\
&= \mathbb{P}(Z_n \leq \frac{\frac{l}{n} - \mu}{\frac{\sigma}{\sqrt{n}}}) \\
&= \mathbb{P}(Z_n \leq (\frac{l}{n} - \mu) \cdot \frac{\sqrt{n}}{\sigma}) \\
&= \Phi((\frac{l}{n} - \mu) \cdot \frac{\sqrt{n}}{\sigma}) \\
\end{align*}

\section{Legge di $X_n$}

Presa una catena di markov omogenea e a stati finiti descritta dalla successione $(X_n)_n$, la densit\`a discreta di una qualche $X_n$ vale

\begin{align*}
p_{X_j}(j) = \sum_{i=1}^n p_{X_k}(i) \pi_{i,j}^{(n-k)}
\end{align*}

assumendo $X_k$ come stato iniziale.

\section{Distribuzione invariante}

Presa una catena di markov descritta dalla matrice $\Pi$ un vettore generico $\pi = (\pi_1, \ldots, \pi_n)$ si dice \emph{distribuzione invariante} se

\begin{align*}
\pi = \pi \Pi
\end{align*}

\end{document}
